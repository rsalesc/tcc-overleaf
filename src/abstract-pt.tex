Identificação de autoria de códigos-fonte é a tarefa de decidir quem é o autor de um programa, dado seu código-fonte. Geralmente, tal tarefa é resolvida partindo de um conjunto de exemplos de códigos previamente coletados, de um conjunto de autores conhecidos. Existem diversas aplicações para métodos de identificação de autoria de códigos, como detecção e atribuição de códigos maliciosos, resolução de conflitos de direitos autorais, detecção de plágio, etc. Assim como em textos em linguagem natural, existem diversas características discriminativas num trecho de código, como nomes de variáveis, estilo de indentação, etc. Algumas dessas características compõem o estilo de programar de um autor. Nesse trabalho, investigamos a atribuição de códigos em C++ baseado no estilo de programar dos autores. Também propomos um modelo baseado em LSTMs que decide se dois códigos-fonte são do mesmo autor, mesmo que as partes envolvidas não sejam conhecidas pelo sistema.