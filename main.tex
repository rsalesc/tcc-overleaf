%% Template para dissertacao/tese na classe UFBAthesis
%% versao 1.0
%% (c) 2005 Paulo G. S. Fonseca
%% (c) 2012 Antonio Terceiro
%% (c) 2014 Christina von Flach
%% www.dcc.ufba.br/~flach/ufbathesis

%% Carrega a classe ufbathesis
%% Opcoes: * Idiomas
%%           pt   - portugues (padrao)
%%           en   - ingles
%%         * Tipo do Texto
%%           bsc  - para monografias de graduacao
%%           msc  - para dissertacoes de mestrado (padrao)
%%           qual - exame de qualificacao de mestrado
%%           prop - exame de qualificacao de doutorado
%%           phd  - para teses de doutorado
%%         * Media
%%           scr  - para versao eletronica (PDF) / consulte o guia do usuario
%%         * Estilo
%%           classic - estilo original a la TAOCP (deprecated) - apesar de deprecated, manter esse.
%%           std     - novo estilo a la CUP (padrao)
%%         * Paginacao
%%           oneside - para impressao em face unica
%%           twoside - para impressao em frente e verso (padrao)

% Atenção: Manter 'classic' na declaracao abaixo:
\documentclass[bsc, en, classic, a4paper]{ufbathesis}

%% Preambulo:
\usepackage[utf8]{inputenc}
\usepackage{graphicx}
\usepackage{lipsum}
\usepackage{hyphenat}
%\usepackage[usenames, dvipsnames, table]{xcolor} %TODO: try to uncomment
\usepackage{booktabs}
\usepackage{pifont}
\usepackage{multirow}
\usepackage{listings} 
\usepackage{colortbl}
\usepackage{xfrac}
%\usepackage[FIGTOPCAP]{subfigure}
\usepackage[printonlyused, withpage]{acronym}

% Custom preamble
\usepackage{cancel}
\usepackage[utf8]{inputenc}
\usepackage{nicefrac}
\usepackage{graphicx}
\usepackage{url}
\usepackage{amsmath}
\usepackage{amsfonts}
%\usepackage{amsthm}
%\usepackage{lineno,hyperref}
\usepackage{algorithm}
\usepackage{algorithmicx}
\usepackage{algpseudocode}
\usepackage{subcaption}
\usepackage{commath}

\newcommand{\etal}{\textit{et~al }}
\newcommand{\refalg}[1]{Algorithm~\ref{alg:#1}}
\newcommand{\reffig}[1]{Figure~\ref{fig:#1}}
\newcommand{\reftab}[1]{Table~\ref{tab:#1}}
\newcommand{\refchap}[1]{chapter~\ref{chap:#1}}
\newcommand{\Refchap}[1]{Chapter~\ref{chap:#1}}
\newcommand{\refsec}[1]{section~\ref{sec:#1}}
\newcommand{\Refsec}[1]{Section~\ref{sec:#1}}
\newcommand{\refsubsec}[1]{subsection~\ref{subsec:#1}}
\newcommand{\Refsubsec}[1]{Subsection~\ref{subsec:#1}}
\newcommand{\procedure}[1]{{\scshape #1}}
\newcommand{\href}[2]{#1} % href completely breaks citations
\newcommand{\vc}[1]{\mathbf{#1}}

% Universidade
\university{Universidade Federal da Bahia}

% Endereco (cidade)
\address{Salvador}

% Instituto ou Centro Academico
\institute{Instituto de Matem\'{a}tica e Estatística}

% Nome da biblioteca - usado na ficha catalografica
\library{Biblioteca Reitor Mac\^{e}do Costa} %TODO: is this correct?

% Programa de pos-graduacao %TODO: is this correct?
\program{Programa de Graduação em Ciência da Computação}

% Area de titulacao
\majorfield{Ci\^{e}ncia da Computa\c{c}\~{a}o}

% Titulo da dissertacao
\title{A deep learning approach to recognize source code authorship}

% Data da defesa
% e.g. \date{19 de fevereiro de 2013}
%TODO: set date
\date{?? de dezembro de 2018}
% e.g. \defenseyear{2013}
\defenseyear{2018}

% Autor
% e.g. \author{Jose da Silva}
\author{Roberto Sales Caldeira}

% Orientador(a)
% Opcao: [f] - para orientador do sexo feminino
% e.g. \adviser[f]{Profa. Dra. Maria Santos}
\adviser{Maurício Pamplona Segundo}

% Orientador(a)
% Opcao: [f] - para orientador do sexo feminino
% e.g. \coadviser{Prof. Dr. Pedro Pedreira}
% Comente se nao ha co-orientador
%\coadviser{Nome Completo do CO-ORIENTADOR}

%% Inicio do documento
\begin{document}

%TODO: change to \frontpage?
\pgcompfrontpage

%% Parte pre-textual
\frontmatter

\presentationpage

%%%%%%%%%%%%%%%%%%%%%%%%%
% Ficha catalografica
%%%%%%%%%%%%%%%%%%%%%%%%%

\authorcitationname{Sales, Roberto} % e.g. Terceiro, Antonio Soares de Azevedo
\advisercitationname{Pamplona, Maurício} % e.g. Chavez, Christina von Flach Garcia
%\coadvisercitationname{Sobrenome, Nome do CO-ORIENTADOR} % e.g. Mendonca, Manoel Gomes de
\catalogtype{Monografia (Graduação)} % e.g. ``Tese (Doutorado)''

\catalogtopics{1. Exact algorithm. 2. Isometric embedding. 3. Maximum norm} % Listar palavras-chave do trabalho para a FICHA CATALOGRAFICA}, por exemplo, ``1. Complexidade Estrutural. 2. Qualidade de Software 3. Engenharia de Software''
\catalogcdd{XXX.XX} % e.g.  XXX.XX (número nesse formato serah dado pela biblioteca)
\catalogcdu{XXX.XX.XXX} % e.g.  XXX.XX.XXX (idem) 
\catalogingsheet

%%%%%%%%%%%%%%%%%%%%%
% Termo de aprovacaoo
%%%%%%%%%%%%%%%%%%%%%

%TODO: set date e banca
\approvalsheet{Salvador, ?? de dezembro de 2018}{
   \comittemember{Prof. Dr. Maurício Pamplona Segundo}{Universidade Federal da Bahia}
   \comittemember{Prof. Dr. Professor 2}{Universidade Federal da Bahia}
   \comittemember{Profa. Dra. Professora 3}{Universidade Federal da Bahia}
   % Para mestrado, apenas 3.
   % \comittemember{Prof. Dr. Professor 4}{Universidade HJKL}
   % \comittemember{Profa. Dra. Professora 5}{Universidade QWERTY}
}

%%%%%%%%%%%%%%%%%%%%%%%%%%%%%%%%%%%%%%%%
% Dedicatoria, Agradecimentos, Epigrafe
%%%%%%%%%%%%%%%%%%%%%%%%%%%%%%%%%%%%%%%%

% Agradecimentos
% Se preferir, crie um arquivo `a parte e o inclua via \include{}
\acknowledgements
%TODO: write acknowledgements
DIGITE OS AGRADECIMENTOS AQUI

% Epigrafe
%TODO: write epigraph
\begin{epigraph}[NOTA]{AUTOR}
DIGITE AQUI A CITACAO
\end{epigraph}

%%%%%%%%%%%%%%%%%%%%%
% Resumo em Portugues
%%%%%%%%%%%%%%%%%%%%%

%TODO: write abstract in Portuguese
\resumo
COLOQUE O RESUMO. Se preferir, crie um arquivo separado e o inclua via comando include.

Para evitar problemas de formato neste template (de uso geral), usamos acentua\c{c}\~{a}o mostrada abaixo. 

\begin{verbatim} 
\c{c} \~{a} \'{a} \^{e} \'{\i}
\end{verbatim} 

N\~{a}o precisa fazer dessa forma, caso use pacotes adequados (latin1, etc.).

%TODO: write keywords in Portuguese
% Palavras-chave do resumo em Portugues
\begin{keywords}
PALAVRAS-CHAVE.
\end{keywords}

%%%%%%%%%%%%%%%%%%%
% Resumo em Ingles
%%%%%%%%%%%%%%%%%%%

\abstract
Source code authorship identification is the task of deciding who is the author of a program given its source code. This is usually based on the analysis of previously collected samples from a set of candidate authors. There are several use cases for such a method, including attribution and detection of malicious codes, copyright infringement incident resolution, plagiarism detection, etc. As with texts in natural language, there are many distinguishing features in a piece code, like variable names, indentation style, etc. Some of these features are part of the coding style of a programmer. In this work, we investigate authorship attribution of C++ source codes based on the coding style of the authors. We also propose an end-to-end deep learning method for deciding if two source codes are from the same author, even if the involved authors are unknown to the system.
% Palavras-chave do resumo em Ingles
\begin{keywords}
software forensics, authorship identification, plagiarism detection
\end{keywords}

%%%%%%%%%%%%%%%%%%%
% Sumario / Indice
%%%%%%%%%%%%%%%%%%%

% Comente para ocultar
\tableofcontents

% Lista de figuras
% Comente para ocultar
\listoffigures

% Lista de tabelas
% Comente para ocultar
%\listoftables


\iffalse
\chapter*{Lista de Siglas}

% Sintaxe da lista de acordo com a documentação do pacote `acronym'
% documentação: http://mirror.unl.edu/ctan/macros/latex/contrib/acronym/acronym.pdf
\begin{acronym}[PGCOMP]
    \acro{PGCOMP}{Programa de Pós-Graduação em Ciência da Computação}
    \acro{CNPq}{Conselho Nacional de Desenvolvimento Centífico e Tecnológico}
\end{acronym}
\fi

%% Parte textual
\mainmatter

%\input{content-sample}
\xchapter{Introduction}{}

Programmers often have to choose between tabs and spaces, between \texttt{while} loops and \texttt{for} loops, between positioning the open bracket in the next line or in the current line, etc. These are choices that can be regarded as their coding styles. Are these choices distinguishable features? In this chapter we will discuss what is style-based source code authorship identification, what challenges it poses, what has been done and what it is useful for. We will also introduce our contributions on the subject.

\section{Biometry and Coding Style}

\section{Motivation}

Throughout this work, we mainly consider the task of an investigator interested in deciding whether two anonymous pieces of code were authored by the same programmer or not. The actual authors of these pieces of code may be unknown to the investigator. Also, these codes may aim to solve different problems, therefore the investigator intends to distinguish them solely based on stylistic features of such pieces.

We also consider easier scenarios where the set of possible authors are known to the investigator and labeled samples from this set are available. % TODO: link to section where these are formulated

We approach these problems from a deep learning perspective, training a deep neural model that can be subsequently used to resolve authorship of source codes and applying it to the different scenarios an investigator can face.

\section{Applications}

Resolving source code authorship has a few real-world applications both in industry and academy. Although we haven't directly studied those, in this section we briefly describe them a few of them. In section (...) we also describe how the experimental formulations we propose in section (...) are related to each of them.

\subsection{Plagiarism Detection}

% alguma referencia q reforce esse paragrafo?
Plagiarism can be generally defined as the unauthorized re-use of the work of another individual. Source code plagiarism is a widespread problem in academic institutions. Checking for plagiarism manually is time-consuming and not extremely effective, becoming impractical as the size of the codebase increases.

Although automatic source code plagiarism detection is a recurring and well-studied problem \cite{plag_survey}, the approaches consolidated by widely used tools such as MOSS \cite{moss}, Sherlock and JPlag \cite{jplag} are mainly based on code similarity metrics which are greatly correlated to the task the code was written to solve.% refer to these tools

For example, let's consider the specific case of \textit{ghostwriting}, where the code the individual claims to be of his authorship was neither written by him nor copied from a colleague, but was actually written by another person (a former student, for example).  It may not be possible to compare the suspicious code with another code by the same author, since the ghostwriter may actually be unknown. On the other hand, if pieces of code of the accused party are available, it's possible to determine if his coding style matches the coding style present in the suspicious code.

Also, an analyst may strongly suspect that a piece of code a programmer claims to be of his authorship is actually not, but have no clue of who the actual author might be. This can be modeled as a binary classification problem where positive samples are other pieces of code of the same author and negative samples are pieces from unrelated programmers. We propose another approach to this problem in (...).

\subsection{Copyright Infringement}

Software forensics is the science of examining source code and binary code in order to identify, preserve, analyze and present facts and opinions about pieces of software. Although it can also be used in civil proceedings, it's most often associated with the investigation of a wide variety of computer crimes, one of which is copyright infringement.

Code correlation analysis plays an important role at copyright disputes. In this case, an analyst has labeled codes from the involved parties and the task of determining if there was infringement or not.

\subsection{Cyber Attack Identification}

Cyber attack identification is a powerful application of software forensics in cyber security. Files left behind by an intruder during a cyber attack may have just enough information for an analyst to identify who the intruder is or to relate such an attack to a previous incident. Therefore, comparing the coding style of the attacker to those of authors of previous attacks and authors of public code repositories is of great interest to the analyst.

\subsection{Exposing Anonymous Programmers}

Although there are many helpful applications of source code authorship identification, systems capable of de-anonymizing programmers pose a threat to those who want to remain anonymous, in special for anonymous open source contributors. In particular, \cite{gitblame} shows how effectively modern methods can identify authorship of small pieces of code from GitHub repositories .There may be good reasons for a programmer to be anonymous, like working in a software a hostile government doesn't like. 

An example of a famous open source anonymous programmer is Bitcoin's creator, Satoshi Nakamoto. For example, if we had a set of labeled codes from programmers that are likely to be Nakamoto, we could try to match their coding style to the early versions of Bitcoin, of course assuming that Nakamoto didn't try to obfuscate his own coding style.

\section{Challenges}

\section{Contribution}
\xchapter{Methodology}{}

In this chapter, we present formulations to the many variants of the source code attribution problem in section \ref{sec:formulations}, we describe how the Codeforces dataset was assembled in section \ref{sec:dataset} and present a top-down approach to how the end-to-end model was developed in section \ref{sec:framework}.

\section{Problem Formulations}\label{sec:formulations}
\section{Datasets}\label{sec:dataset}

The first step to develop an effective deep learning model is to gather enough training data. In this work, we decided to work with C++ source codes written in a laboratory environment -- we assume the whole code is written by the author under no external style enforcement such as a style guide.

\subsection{Google Code Jam}

Although there are many public C++ laboratory datasets, the Google Code Jam\footnote{\url{https://codingcompetitions.withgoogle.com/codejam}} dataset \cite{caliskan_2015} is probably the biggest of them all. Samples from this dataset are collected from previous editions of Google Code Jam, an annual programming competition held by Google. In this competition, participants are given algorithmic tasks to be solved in a limited amount of time. As such, it's very likely that code written by a participant manifests his own coding style.

Google Code Jam holds nearly 10 rounds every year. Most of these rounds are eliminatory. Thus, the availability of samples from less experienced participants is expected to be low. If we want to build a balanced training set not biased by the way experienced participants code, we are limited by the small amount of code less experienced participants wrote.

Although this dataset was not extensively used throughout the development phase, it was a reference for the Codeforces dataset introduced in section \ref{sec:codeforces}.

\subsection{Codeforces}\label{sec:codeforces}

Codeforces\footnote{\url{http://codeforces.com}} is a website specialized in holding online programming contests. Contest format is similar to Google Code Jam's, but they are not eliminatory. Thus, we are able to find a lot of samples from both non-experienced and experienced users.

We wrote a Python script that receives target constraints for the dataset and scrapes Codeforces for samples. Using this script, we assembled a balanced dataset with more than 30,000 samples from nearly 2,000 authors, meaning that we have around 15 samples per author. This dataset was packaged and made public\footnote{link-pro-dataset}.
% TODO: more details on filters
% TODO: add link to dataset

\section{End-to-End Deep Learning Framework}\label{sec:framework}
\subsection{Coding Style Descriptor}

The performance of machine learning methods is heavily affected by the choice of data representation. Thus, much of the effort of the machine learning community has been put into developing algorithms that transform otherwise unmanageable data into representations that can be effectively used by learning methods \cite{representation_learning}.

A Coding Style Descriptor (hereon referred simply as \textit{style descriptor}) is a $\mathbb{R}^n$ latent representation of a source code that captures its stylistic features. Ideally, style descriptors should encode everything a machine learning model needs to solve the problems posed in section \ref{sec:formulations}. Thus, we can build simpler classifiers for these problems if we are able to build good latent representations for source codes.

Deep feed-forward networks are a natural approach to representation learning. In the remainder of this chapter, we will mainly study deep learning techniques to generate style descriptors from source codes.

\subsection{Preprocessing}

\subsection{Models}
\subsubsection{Char-Level CNNs}
\paragraph*{Network Architecture}
\subsubsection{Hierarchical LSTMs}
\paragraph*{Network Architecture}
\subsection{Optimization}
\xchapter{Evaluation}{}\label{chap:evaluation}

% TODO: regenerate this table
\begin{table}[htbp]
	\centering
	\begin{tabular}{c|l}
		\hline
		\textbf{Parameter}           & \multicolumn{1}{c}{\textbf{Value}} \\ \hline
		maximum line length   & (...)                            \\ \hline
		maximum number of lines   & (...)                            \\ \hline
		$d_c$, char embedding size   & (...)                            \\ \hline
		$d_l$, line descriptor size  & (...)                            \\ \hline
		$d$, style descriptor size & (...)                            \\ \hline
		char-level LSTM hidden units & (...)                            \\ \hline
		line-level LSTM hidden units & (...)                            \\ \hline
		fully-connected layer units  & (...)                            \\ \hline
	\end{tabular}
	\caption{Values for the hyperparameters of the LSTM-based model.}
	\label{tab:lstm_hyper}
\end{table}
\xchapter{Conclusion}{}

This work offers a study on the application of deep learning methods to the source code authorship attribution problem. In particular, we have proposed a LSTM-based model that generates descriptors which preserve stylistic similarities between source codes. Moreover, we have shown experiments supporting that stylistic features are discriminative. 

We evaluated our model in the problem of deciding if two codes were written by the same person and achieved 11.6 \% of EER. In the one-to-many identification problem, we achieved 74.8\% accuracy, 20\% less than state-of-the-art. Although we were not able to improve this result, we believe state-of-the-art deep learning methods can be successfully applied to the source code attribution problem, and that further improvements in the area will be related to that.

In the future, we hope to modify our models to support variable-length code. We also want to generate meaningful style descriptors from abstract syntax trees to improve the accuracy of our methods. Moreover, we want to be able to classify small pieces of code, like those of Git repositories.

%% Parte pos-textual
\backmatter

% Bibliografia
% É aconselhável utilizar o BibTeX a partir de um arquivo, digamos "biblio.bib".
% Para ajuda na criação do arquivo .bib e utilização do BibTeX, recorra ao
% BibTeXpress em www.cin.ufpe.br/~paguso/bibtexpress
\bibliographystyle{abntex2-alf}
\bibliography{src/biblio}

% Apendices
% Comente se naoo houver apendices
\iffalse
\appendix

\xchapter{Exemplo de Apêndice}{} %sem preambulo
\lipsum
\fi
% Eh aconselhavel criar cada apendice em um arquivo separado, digamos
% "apendice1.tex", "apendice.tex", ... "apendiceM.tex" e depois
% inclui--los com:
% \include{apendice1}
% \include{apendice2}
% ...
% \include{apendiceM}

%% Fim do documento
\end{document}
%------------------------------------------------------------------------------------------%